\section{5G technologies}

\subsection{Context}

The course of 5G technologies is taught by Etienne Sicard. The main objective of this course is to discover a lot of actual topics regarding cellular network. This course was taught using reverse pedagogy which mean that the content of the course was mainly  composed of presentation done by students. As the industry of cellular network evolve at a fast pace, it was really interesting to discuss about such a recent and emerging topic. Indeed 5G is a big change in cellular network as it is the first generation to introduce software defined radio (SDR) and microservices. This totally changed the architecture of cellular networks.

\subsection{Technical aspects}

During this course we could discuss about many varied subject. I had the opportunity to do with Aude Jean-Baptiste another student in ISS a presentation on the 5G core architecture with all the different microservices. They offer much more flexibility as it would be easier for telecom companies to fit the needs of special zone. Indeed as everything in 5G is transitioned to software, they would be able to deploy other microservices to adapt to the charge of the network. For example, in urban zones, it would be possible to deploy a same service multiple time and use a load balancer to increase the capacity of the network. 
\smallskip

Another thing that microservices and SDR brings is the ease of evolution. Indeed, switching to a hardwared infratstructure to a more softwared infrastructure is good for evolution as it is a lot cheaper and easier to update a software than to change hardware for future implementation. The idea behind 5G is that we could keep the same hardware infrastructure and deploy update of 5G with only software and even deploy new generations of cellular networks like 6G without needing to change anything.
\smallskip

Finally another thing that 5G brings and which is in my opinion the most important change is the concept of slicing. Before 5G every devices used the same network. However, we have a lot of different categories of devices which have all different needs. Indeed, IoT devices for example generally requires low bandwidth but energy efficiency and massive capacity as a lot of devices need to connect to the network which makes previous cellular network not very well suited for IoT devices. However in 5G technologies, the concept of slice introduces the division of a physical network in logical network which are suited for specific category of devices. We can distinguish 3 different slices. URLLC (Ultra Reliable Low Latency Communication is suited for critical devices that needs guaranties on liability and latency for example connected cars. eMBB (enhanced Mobile BroadBand) is the classical slice dedicated for smartphone users. Finally mMTC (massive Machine Type Communication) is dedicated for sensors and IoT devices.
\smallskip

In this course we also had some presentation about other subjects linked to 5G like satellites, the semiconductor industry or even V2X (Vehicle to Everything). I will not go in detail on these subjects for length reason and because they are only linked to 5G and are not part of the standard. 

\subsection{Skill Matrix}

\begin{itemize}
    \item Understand and master the new mobile networks technologies
\end{itemize}

For this course I think that I master the 5G architecture with its microservices. However, I feel that we did not spent enough time talking about 5g hardware and especially SDR (Software Defined Radio) so I cannot say that I master these aspects of 5G technologies.

\subsection{Analysis} \label{Analysis_5G}

This course was really interesting as it was linked to actual subjects. However, I do not particularly appreciate courses in reverse pedagogy as I think that we are more focused on our own work which can makes the comprehension of other's presentation harder are we are focused on improving our presentation. I think that a better balance between lecture by the teacher and presentation would be better. In the end, I  only have high level knowledge on subjects presented by other while if the teacher would have done lecture on it, we could have dive more deeply in these subjects.
\smallskip

I also thought that for a course called 5G technologies we did not focused enough on the 5G and 6G standards while they are the present and future of cellular network which would be really interesting to focus on especially as these standards diverge a lot from the previous generations of cellular networks.