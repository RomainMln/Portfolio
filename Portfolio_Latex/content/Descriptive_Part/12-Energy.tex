\section{Energy for Connected Objects}

\subsection{Context}

Energy for Connected Object is a course taught by Gaël Loubet. This course is an introduction to how to power an embedded devices. During this course we saw the 3 different methods on how to power a device as well as all the different kind of energy that we can harvest in an embedded system. It was composed of few theoretical classes followed by two labs on turning on a LED through wireless transferred power.

\subsection{Technical Aspects}

During this course we saw the different methods to power an embedded system as well as the different energy sources available to harvest or to use for wireless power transfer.
\\\par

There are three different method to power an embedded system. The first and easiest one is to power directly the device. To do so we put a battery in the embedded system that from which the device can take power when it needs it. The major pro for this method is the ease of implementation and decide on the dimensions. Indeed, putting a battery is very easy and it allows to plan precisely the interval of transmission and the lifetime of the device as long as we can define the consumption of the embedded device. The major cons is that we need to charge or change the battery of the device  regularly which can be difficult in some cases (device in the concrete for example).
\par 
We can also use this method without needing a battery by using wireless power transfer which is named direct consumption. This remove the cons of the lifetime of the device but create another one which is that we need an external energy source that power the device. As the yield of that kind of system is rarely 1, then overall we lose some energy.With this method we do not store any exceeding energy.
\\\par

The second method is to only use energy harvesting to power the system by storing the collected energy until it reaches enough energy to power the device. It can harvest the energy present in the environment but also the energy from a wireless power transfer source. The pros and cons of this method are the opposite of the first one. Indeed, this method can lead to a longer lifetime for the devices but is harder to define the interval of transmission and it can also be harder to implement than simply putting a battery. 
\\\par

The last method is a mix of both methods which is to directly power the device. With that method, we power the device directly by harvesting the ambient energy or by using a wireless power transfer source but we store the exceeding energy in order to use it when there is not enough energy to directly power the device. To illustrate this methods we can take the example of a device powered by solar power during the day and that uses the store exceeding energy during the night.
\\\par

There are a lot of different energy in the environment that we can use to power a device. We can use electromagnetic fields, mechanical energy (wind turbines for example), solar energy, thermal energy (using the difference of temperature). To power a device we can also use primary and secondary batteries.

\subsection{Skill Matrix}

\begin{itemize}
    \item Know how to harvest/transfer, store and manage power for connected objects, and how to increase the power efficiency
    \item Be able to optimize the power consumption of connected objects
    \item Be able to design and implement an energy autonomous and battery-free connected object
\end{itemize}

Through this course, I could understand all the concepts of choosing and optimizing the power management for an IoT device. As we saw all different kinds of powering methods and energy sources, I think that I master the skill of knowing how to harvest/transfer, store and manage power for connected objects. As I do not have a strong background in electronics or physics, I do no think that I master the optimization of  power consumption and design of a battery free connected object but I definitively understood the basic concepts.

\subsection{Analysis}

When I started this course I did not really enjoyed it as it was something that has nothing to do with my professional goal to become a network engineer. However, I really enjoyed this course and could learn a lot from it. Indeed, this course was really well defined to only be an introduction to the concepts of powering an embedded device. Only few physics or electronic background was required and it was very enjoyable to follow. Through this good organization I could really get the knowledge and skills that the teacher wanted to taught through this course.
\\\par

However, for the negative part, I think that the labs were a bit chaotic as nothing seemed to be working as intented. Overall, we could see the point of the lab and get through the manipulations but I think that it suffers from technical problems with the PCB done at INSA. As it is only the first year that these PCB were made, it is perfectly understandable that it has some problems. 