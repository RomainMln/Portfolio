\section{Internship at TNO in the netherlands}

\subsection{Environment and context}
The Netherlands Organisation of applied science research (TNO) is a Dutch research laboratory
that has 24 sites in the Netherlands, 1 in Belgium and 1 in Japan. It is composed of 6 mains units :
\smallskip

\begin{itemize}
    \item Mobility \& Built Environment
    \item Energy \& Material Transition
    \item \textbf{Defense, safety \& security}
    \item Healthy living  \& work
    \item High Tech Industry
    \item ICT, Strategy Policy
\end{itemize}
\smallskip

I was part of the Defense, safety and security. I was assigned to the Softanet Project which
objective is to develop the interaction between federal Software Defined Network (SDN) in order to
establish end to end slicing allowing us to give guaranties (on the bandwidth, delay and jitter) to
the end users.
\smallskip

I chose to do my 4th year summer internship at TNO because I was really interested in having
an international working experience to see the differences with a french company. In my opinion,
having this vision of the work place in two different countries is really useful for an engineer because
it allows discover new way of thinking, working and dealing with problems and in our future work
implementing the best from the two worlds to get a better work environment and an optimal
efficiency.

\subsection{My function}


My role in the softanet project was to design and deploy monitoring functions inside a SDN.
The need here is to know the state of our own network with how much resources are available.
Firstly, it would allow an administrator to check that everything is working, to be warned when a
link is down ... Secondly, as it is a SDN, we could automatically accept or refuse new connections
to the network based on the state of the network.


\subsection{Technical solutions}

To implement this monitoring functions, I used prometheus which is a software that allows to
"scrape" different targets to collect metrics from these targets. I used the snmp exporter as well as
the sflow collector exporter and the node exporter that exposes more metrics from the devices for
prometheus to scratch. These exporters are network oriented and allowed me to pull new metrics
like the actual and max bandwidth of each link, if they are up and other information like the number
of packet loss.
\smallskip

As we are in a SDN context, I also used the network controller, here OpenDayLigh (ODL) which
is an open source network controller. By making request to the northbound API of the controller,
I could get the topology of the network in a JSON file. By parsing the topology and mixing it with
the metrics that I got from prometheus, I could construct a graph with the topology of the network
and the state of each link. To do that, I used Grafana. It is a software that allows us to construct
graphs based on metrics that we provide. It is a relevant solution in my case because it has a pro-
metheus and JSON API plugin that allows me to import the appropriate data sources for my graph.
Finally, my last mission was to deploy all these monitoring functions dynamically through the
use of scripting. To do that, I used Ansible which is a scripting software that by connecting to
targets through ssh can manage hosts. That way, I could install and configure automatically all the
software needed on the different targets


\subsection{Problems faced and solved}

During my internship, I faced some problems. The first main challenge was that when I arrived
I did not have any experience in monitoring networks. I knew some protocols that I studied in
my 4th year communicating systems like SNMP (Simple Network Management Protocol) but this
protocol is really limited. To solve this problem, I spent my first month looking online on all the
different existing solution to monitor networks. I also had the opportunity to talk a lot to expert
in the company that knew some software like prometheus or grafana or sflow so I could find a first
track to begin my searches.
\smallskip

For the infrastructure of deployment, I had to use TNO private cloud with Openstack to deploy
Virtual Machines (VMs) for implementing my topology. I did not had at that time any knowledge
on how cloud worked so I had to learn how to use it efficiently.

\subsection{Skills learned}

From a technical point of view, I could learn a lot in network by doing my researches but mainly
by talking to experts in networks that had a lot of experience in this domain. I learned a lot about
monitoring and how to implement it. I also could learn and manipulate some SDN which is a concept
that we see in 5ISS. I also could discover scripting deployment with ansible which is something that
I really enjoyed doing.
\smallskip

From a personal point of view, this internship help me to confirm my professional project.
Indeed, I am now sure that I want to be a networking engineer as it is something I really enjoyed.
I also could develop my English, I had the opportunity to discuss with people from all around the
world working at the company which helped me to discover new cultures, way of thinking and way
of working which is really enriching.


\subsection{Analysis}

As mentioned before this experience was very enriching and I could learn a lot in networking.
However, for the drawbacks of this internship, I would have liked to be part of a group working
on the same subject because it would have gave me the opportunity to interact more with other
employees and to fully discover how to work in team in a company.