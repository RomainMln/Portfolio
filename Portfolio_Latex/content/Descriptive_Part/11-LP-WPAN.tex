\section{Low Power Wireless Personal Area Network  (LP-WPAN)}

\subsection{Context}

Low Power Wireless Personal Area Network (LP-WPAN) is taught by Slim Abdelatif. It is a very short course composed only of lectures. The objective of this course is to discover a specific TCP/IP protocol stack for LP-WPAN. With the training of ISS centered on IoT, it is very interesting to see an example of implementation a sensor network protocol stack.

\subsection{Technical Aspects}

As this course is very short, we did not have a lot of time to see a wide range of different technical skills. In addition to that a first part of this course was dedicated to explain basic concepts about network and telecommunication like the concept of a scrambled network (interference). 
\\\par
During this course, we saw an example of TCP/IP protocol stack for LP-WPAN. This stack used the 802.15.4 norm which is a norm for MAC and physical layer. It is notably used in protocols like Zigbee. Over this layer we use IPv6 with a intermediate layer to reduce the size of a packet which is 6LoWPAN.

\subsection{Skill Matrix}

We were not provided the skill matrix for this course, so I would try to analyses the skills that I could developed. This course was in my opinion an extension of the WSN course in the way that the same subjects are approached but using a different point of view. This course gave us a good example of a TCP/IP stack  for a LP-WPAN. One of the main skill was to understand the benefits and cons of using standards protocol based on IP. Another important skill that we developed during this course was understanding how to integrate IPv6 in a low power use case with 6LOWPAN.
Through the theoretical classes that we had during this course and with my networking background, I think that I was able to master the different skills that were aimed by this course.

\subsection{Analysis}

This course was really interesting as it explains different technical choices for protocols in a LP-WPAN context. However this course was really short and most of the lectures were dedicated to explaining network and telecommunication concepts that we, as students from networking background, already saw. In the end, this course was more targeted for student from computer science, electronics and physics background. 
\\\par

During this course, there was a lot of overlaps with the WSN course. Indeed, the teacher presented the IEEE 802.15.4 norm which is a norm that we already presented with Zigbee in the first assignment for WSN. I think it is unfortunate to have redundant information on a course that is so short.
\\\par
However, during this course I really appreciated the fact that the teacher explained his own visions on how to implement a wireless solution for LP wireless protocols. Indeed, during the WSN course, the teacher defended the fact that it is better for LP protocols to design ourselves a protocol that specificaly fit our needs. During the LP-WPAN class the teacher said that even if he knew the other positions from the other teachers at INSA, he thought that it was easier to use an already existing stack based on IP for integration purposes. 
\par
Without taking position on who is wright and who is wrong, I found really interesting to have diverging opinions ton the same topic from the teacher as I makes us think by ourselves on the argument and not taking what a teacher says as absolute truth. It made us develop our critical mind which is really important for engineers. 