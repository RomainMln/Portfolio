\section{Lab at AIME}

\subsection{Context}

As ISS is focus on IoT, we use a lot of different sensors. Indeed, to make a smart device and able to take decisions, it first need to analyse its environment. We often uses the final product without realizing how do we make and how much does a single sensor cost. During a week at the AIME (Atelier Interuniversitaire de Micro-nano Électronique), we had the opportunity to create a gas sensor from a simple half made wafer.

\subsection{Technical Aspects}

This course was mainly focused on chemistry and physics. As it is not my main skills and it is not something that I want to do professionally, I will not develop a lot this course. 
\\\par

As mentioned in the context, this labs were only an introduction to the creation of sensors. We created in parallel, the chemical solution containing nanoparticles that will detect the gas and the wafer on which we applied the chemical solution. After this step, we did a dielectrophoresis to place the nanoparticles on the wafer. A dielectrophoresus is a physical manipulation that consists of applying a electrical field on the wafer while placing the nanoparticles on it in order to stick them between the interdigitated comb. That way, when a gas is detected, the resistance of the sensor drop because the electrical field goes through the nanoparticles instead of the resistance. 

\subsection{Skill Matrix}

\begin{itemize}
    \item Be able to manufacture a nano-particles sensor using micro-electronics tools: chemical synthesis, assembly, testing 
\end{itemize}

As I do not come from a physical or chemist background. I cannot say that I master the conception and manufacturing of a nano-particle sensor but I can say that through this course I was able to have a better understanding of the different concepts.

\subsection{Analysis}

This course was really interesting as it helped us realize all the process that hides behind the creation of sensors. The instructions were really clear and beginner friendly which made these labs a good introduction to the world of physics and chemistry. The teachers were pedagogue and helped us to understand every processes of the synthesis and creation of the wafer.
\\\par
However, this course suffered from coordination problem between the ISS teachers and the ones from the physical department which resulted on these labs being placed during the holidays which hardened the understanding of the concepts.