\section{Innovative Project}

\subsection{Context}

During this year in ISS, we have the opportunity to work all over the semester on an innovative project. We work in groups of 4 or 5 students on a project we chose among several subjects which some are propose by industrial companies. I worked with Aude Jean-Baptiste, Emily Holmes and Cyprien Heusse on a project proposed by Continental. The project is entitled Advanced Cyclist Assistance System (ACAS). The main objective is to increase cyclist safety as more and more people die on the road. \par
To do so, we have to both detect a danger for the cyclist and to warn the cyclist of the incoming danger.
\\\par
For the detection part, it uses a camera placed on the back of the bike which is connected to a jetson board which is an embedded system that analyzes the images taken by the camera. A pre-trained IA is responsible to detect the danger. This step of detection has been done by Continental. On our side we are responsible to create the warning system.
\\\par
For the warning system, we needed to alert the cyclists without perturbing them on the road. To do so, we firstly developed an android application that connects to the jetson board through Wifi in order to get notifications in case of a danger. When it happens, the application display the video in real time of the rear view of the bike with the detected danger circled. That way, the cyclists can see the identified danger without needing to turn themselves. Another system designed to alert the cyclists is a vest that has integrated haptic and bluetooth modules in order for the jetson board to send orders to vibrate to the vest to alert the cyclists.
\\\par

As some part of this project are confidential, the report is not available on the github.

\subsection{Technical Aspects}

During this project we had the opportunity do develop a lot of different technical skills. As students coming from a computer science and more precisely a network background, I had the opportunity to use my knowledge in networking. Indeed, in order to work efficiently on the jetson, we wanted to use SSH to work with our own machines. To do so, we configured a DHCP server on the jetson that would automatically give us an IP address when we would connect our pc through ethernet. 
\\\par
In addition to that, for this project, we had the communication protocols between the jetson and the phone as well as between the jetson and the vest. With my knowledge and considering the requirements for the project (bandwidth to send video, QoS ...) we  decided to use Wifi for the communication between the jetson and the phone and Bluetooth between the jetson and the vest.
\\\par
Finally, I was responsible to create the android application following a design that Cyprien and Emily did. As I never developed an application neither I ever used Kotlin, this was very challenging. I had to learn a lot about the Kotlin language and how to integrate video streaming into an application. 

\subsection{Skill Matrix}

\begin{itemize}
    \item Analyse a real-life problem
    \item Suggest a technological solution to a problem
    \item Implement a prototype to solve the problem
    \item Present and debate (in English) the technical choice made
    \item Produce a report (in English) for the developed project 
\end{itemize}

For this project, the skills aimed are really well defined. Indeed, as this project was proposed by a company it is a real-life problem that we have to face and analyses. The variety of background and by extension skills in our team ease and make interesting the sugestion and implementation of a whole technical solution for our problem. And finaly, as the report and the defense of the project are done in English, this project help  us to develop our presentation, debating and writing skills in English. To conclude, I think that through this project I was able to master all the skills present in the matrix.

\subsection{Analysis}

This project was really enriching and helped us develop a broad variety of new skills. \\\par

The first thing that I really appreciated during this project is the fact that we mixed between students from computer science and electronics backgrounds. That way we, could discover how to work in group, how to split tasks and time scheduled when everyone has different skills. This really felt like a more concrete project than anyone that we previously had at INSA.  \\\par

I also really appreciated the help from Continental. Indeed, it really felt like they were invested in the project as they stayed available and responded to our needs and problems very quickly. They also understood the fact that this is a project done by students with very limited time over the semester and did not expect a fully working, ready to sell product at the end of the project unlike another company that proposed a subject. They only expected a proof of concept with a demonstration at the end which we delivered as well as the next steps that has to be worked on.