\section{Emerging Networks}

\subsection{Context}

Emerging networks is a course taught by Slim Abdelatif. It is a short course of only few hours about new paradigm for networks. We discover this paradigms during the theoretical class and have the opportunity to manipulate Software Defined Network (SDN) during practical courses. 
\\\par

Unfortunately, as all the labs are placed after the deadline of the portfolio report, I will not be able to talk about it in this part which shorten a lot the content of this part.

\subsection{Technical Aspects}

In this course, we studied the new network paradigms. The first one that we studied in detail is SDN. In this paradigm we remove all intelligence from the network devices (switches and routers) and centralize it in a single point which is the network controller. The objective behind this decision is for a network administrator to be able to take decisions and pilot the network by interacting with the network controller using classical programming. It also allows to handle packet more precisely and implementing behaviour of the network on demand and not being dependant of the functionalities proposed by the device manufacturer.
\\\par

The other paradigm that we saw is the  Locator/Identifier Separation Protocol (LISP) protocol it consists of attributing both IP address to a device. One used for the localisation and the other to identify uniquely the device. This allows mobility of node as only one address change and not the other. That way, to contact another node, we use the identifier node that never change, after that a protocol similar to DNS is run to find the localisation address to reach the node. 
\par
However this is only possible if the edge router of the network is compatible with the LISP protocol as it is this one that will execute the translation.

\subsection{Skill Matrix}

\begin{itemize}
    \item Understand  and master the fundamentals of emerging network paradigms applied to IoT
\end{itemize}

As a student coming from a networking background, I already had solid bases on the concepts that were approached in this course. Indeed, we saw the SDN paradigm during my 4th year internship at TNO as well as the Master REOC that I will talk about in the section of this part. As for LISP, we already saw similar concepts during the 4th year course of IPv6 that can handle the mobility of nodes. So I easily mastered the fundamental concept that were approached during this course.

\subsection{Analysis}

As a student coming from a networking background I really appreciate this course because it deals with what I think is the future of network. Indeed, I think that the flexibility that offers SDN will allow network administrator to implement the functionalities that they want. In a general way, I think that every field tend to softwarization with its flexibility. To illustrate this  point I can take the example of the 5G standard that has SDR (Software Defined Radio) as well as microservices for its core architecture. 
\\\par
However, in the case of ISS, I do not really see the point of approaching these subjects as they SDN is suited for the core network and not really suited for IoT networks. This is a point that the teacher talked about multiple times during the lecture. LISP can be interesting for IoT as device in this context can be mobile.