\part{Introduction}

\section{Background}

I am Romain Moulin a student at the French National Institute of Applied Science of Toulouse (INSA Toulouse). During my studies, I specialized in computer science and more precisely in the networking and telecommunication pathway. For my 5th and last year, in joined the Innovative Smart System (ISS) class which is opened to students from computer science, electronics or even physics backgrounds. It is focused of Internet of Things (IoT) as this field require skills specific from each three background. I decided to do this as is was in my opinion, a pathway that got some networking which is the field I want to work in and also because I like the idea of being mixed with other students, all having their own skills.

\section{Presentation of the Portfolio}

As an INSA student, we had the opportunity to follow many different courses on various topics. As this courses are often composed of labs and project, we found interesting to keep a track of all the work accomplished as well as the different skills.
\\\par 
This document is the record of all the different project in the ISS program as well as some personnal of professional project that are relevant regarding my professional perspectives. It is composed of a brief introduction and presentation of the skill matrix of the 5ISS's projects. Then in the second part, I will describe and analyse the projects that I did. Finally, in a last part, I will take a step back on all my curriculum and give some feedbacks about the formation.
\\ \\
\textbf{NB: }This document is part of a \href{https://github.com/RomainMln/Portfolio}{github} repository that will contain all the reports for the different projects as well as my CV.

\section{The skill matrix}

\begin{center}
    \begin{tabular}{|L{1.75cm}|L{7.5cm}|C{1.5cm}|L{3.75cm}|}
        \hline % Ligne horizontale de séparation
        \rowcolor{black!15!white}\textbf{Intitulé} & \textbf{Organisme} & \textbf{Année} & \textbf{Durée (en heures)} \\
        \hline % Ligne horizontale de séparation
        ISS & Institut National des Sciences Appliquées & 2019 & $+\infty$ \\
        \hline % Ligne horizontale de séparation
    \end{tabular}
\end{center}
