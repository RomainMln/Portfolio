\sectionnn{Introduction}

In the race of innovation in the Internet of Things, security is often left behind. However, as the Internet of Things is massively developing, the attack possibilities are growing exponentially. Therefore, connected objects are vulnerable to direct attacks towards the objects themselves, but they are also vulnerable to becoming a mean of attack, by becoming part of a botnet. Botnets can then be used to perform large scale attacks such as DDoS.
\smallskip

Many objects in the IoT are vulnerable firstly because of their configuration. Let’s take the example of a small business owner wanting to secure his premises. He will likely buy a “plug and play” connected security camera (CCTV) in order to keep an eye on his shop from his home. It is highly probable he will not try to change the basic configuration of the object, such as the passwords.
\smallskip

Unfortunately, these IoT devices often come with weak default username and password (such as “login: admin, password: admin”) and therefore are vulnerable to dictionary attacks. We can take as an example the Mirai botnet \cite{mirai_botnet}, which scans the Internet looking for IoT devices with default username and password using dictionary attacks over telnet. Once a new vulnerable device has been discovered, the Mirai binary is injected in the device and it joins the Mirai botnet, making it stronger and capable of finding new devices faster. These kinds of botnets are then used to launch bigger attacks such as distributed denial of service (DDoS) on more robust structures.
\smallskip

Thus, one of the biggest challenges in the Internet of Things is to be able to secure devices and, therefore,  to detect when and how a device is being attacked. In this study, we will investigate the existing research material on intrusion detection systems (IDS) used to detect if a connected object has been compromised. The IDS will use physical measures as well as network monitoring, but, in this study, we will particularly develop the case of  physical measures.








