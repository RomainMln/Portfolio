\part{Conclusion}

\section{Summary of my year in ISS}

This last year in ISS was very disappointing for me. Indeed, I do not feel like I learnt a lot of new concepts during the year or at least concepts related to my professional perspectives of being a network engineer. Indeed, as all the courses needs to be accessible for everyone, we already saw most of the content of the courses related to computer science or networking. As for other courses, I learnt few things on electronics, physics and chemistry but even there, not as much as I would have imagined. \par
As I already illustrated it in my analysis of the different courses, I think that this pathway suffers a lot from organizational problems that make some courses redundant or not pleasant to follow. \\\par

As a positive thing that I get from this year, I could meet a lot of different students from different background that I really appreciated discussing with and discovering their fields of expertise. I also think that with all the different subject that we saw, we have a much more broader view of every things that comes with the creation of sensors and the concept of IoT in general. \par
Another positive point about ISS is that it gave me the opportunity to do an additional master specialized in Networking and Telecommunications which is something that is very valuable for my CV and helped me develop my knowledge regarding networks. 
\\\par
To conclude, I think that if I had to do again my choice of pathway for my 5th year, I would not have chose ISS and would have go in SDBD with the minor SDCI (networking). That class seems more suited for me as I would have studied much more networking which is really important for me as I want to work in that field.